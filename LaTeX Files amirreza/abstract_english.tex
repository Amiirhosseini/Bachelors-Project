\addcontentsline{toc}{chapter}{چکیده انگلیسی}
\thispagestyle{empty}

\begin{latin}
\begin{center}

{\Huge Edge Computing Empowers Federated Learning in Resource-Constrained IoT Networks}

\vspace{1cm}

{\LARGE{Amirreza Hosseini}}

\vspace{0.2cm}

{\small amirreza.hosseini@ec.iut.ac.ir}

\vspace{0.5cm}

September 2023

\vspace{0.5cm}

Department of Electrical and Computer Engineering

\vspace{0.2cm}

Isfahan University of Technology, Isfahan 84156-83111, Iran

\vspace{0.2cm}

Degree: B.Sc. \hspace*{3cm} Language: Farsi

\vspace{1cm}

{\small\textbf{Supervisor: Dr. Amir Khorsandi (khorsandi@iut.ac.ir)}}
\end{center}
~\vfill



\noindent\textbf{Abstract}

\begin{small}
\baselineskip=0.6cm
Artificial Intelligence (AI) has rapidly expanded into various fields in recent years, becoming a fundamental technology in numerous applications. To meet the diverse needs of these applications, various approaches and methods have been proposed to achieve the infrastructure related to AI technology. One of these methods is the Machine Learning (ML) approach for privacy preservation, known as Federated Learning (FL). FL enables the development of applications that perform analysis and AI-based services on sensitive user data while complying with strict privacy regulations in various domains, including healthcare, finance, and transportation, particularly the Internet of Things (IoT).
It is worth noting that FL can be executed on a large number of heterogeneous final devices, which are typically distributed and have limited processing power and energy resources. This presents a significant challenge that can hinder the effective, sustainable, and scalable execution of FL.

To address these challenges, recent research has proposed the use of edge computing in FL. Edge computing involves using the storage, communication, and computing capabilities of edge servers located near end devices to reduce system latency. Performing primary computations and preprocessing close to the data source can reduce the amount of traffic sent to the main server and decrease access delays. This approach can also be effective in improving the performance of FL, which is the main focus of this project.

Therefore, the objective of this project is to propose a structure that provides a suitable platform for developing FL in networks with edge computing capabilities. The proposed structure aims to address the challenges of executing FL on heterogeneous and resource-constrained devices by leveraging the capabilities of edge servers. The resulting system improves the performance and scalability of FL while complying with strict privacy regulations.

\end{small}

\vspace{0.5 cm}

\noindent \textbf{Key Words}: Federated learning, Internet of Things, Edge computing, Machine Learning, Artificial Intelligence

\end{latin}