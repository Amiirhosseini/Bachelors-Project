%*************************************************
% In this file the abstract is typeset.
% Make changes accordingly.
%*************************************************

\addcontentsline{toc}{section}{چکیده}
\newgeometry{left=2.5cm,right=3cm,top=3cm,bottom=2.5cm,includehead=false,headsep=1cm,footnotesep=.5cm}
\setcounter{page}{1}
\pagenumbering{arabic}						% شماره صفحات با عدد
\thispagestyle{empty}

~\vfill

\subsection*{چکیده}
\begin{small}
\baselineskip=0.7cm
امروزه، کاربرد هوش مصنوعی به سرعت در حوزه های مختلف گسترش پیدا کرده است. با توجه به نیازهای متفاوت در کاربردهای مجزا، تا کنون رویکردها و روش‌های گوناگونی برای تحقق زیرساخت‌های مربوط به فناوری هوش مصنوعی ارائه شده‌اند. یکی از این موارد، رویکرد یادگیری ماشینی برای حفظ حریم خصوصی است که به عنوان یادگیری فدرال شناخته می‌شود. با توجه به مقررات سختگیرانه حریم خصوصی در حوزه‌های مختلف از جمله پزشکی، صنایع و غیره، یادگیری فدرال امکان توسعه برنامه‌هایی را فراهم می‌کند که با استفاده از داده‌های حساس کاربر، به تجزیه و تحلیل اطلاعات و ارائه خدمات مبتنی بر هوش مصنوعی و یادگیری ماشین می‌پردازند. این موضوع می‌تواند زمینه‌ساز یک تحول اساسی در ارائه خدمات در امور مالی، مراقبت‌های بهداشتی، صنایع حمل و نقل و به صورت کلی اینترنت اشیاء باشد. البته نکته حائز اهمیت این است که یادگیری فدرال ممکن است بر روی تعداد بسیار زیادی از دستگاه‌های نهایی ناهمگون و به صورت توزیع شده اجرا گردد که بسیاری از این دستگاه‌ها توان پردازشی و دسترسی به منابع انرژی محدود دارند. این یک چالش اساسی بر سر راه اجرای موثر، پایدار و مقیاس‌پذیر یادگیری خواهد بود.


در این پروژه، با توجه به ظرفیت پردازش لبه، ساختاری پیشنهاد می‌شود تا بتواند بستر مناسبی برای توسعه یادگیری فدرال در شبکه‌هایی با کاربرد در اینترنت اشیاء را فراهم سازد. به صورت کلی در پردازش لبه، از ظرفیت‌های ذخیره‌سازی، ارتباطی و محاسباتی سرورهای لبه که نزدیک به دستگاه‌های پایانی مستقر شده‌اند، برای کاهش تاخیر سیستم استفاده می‌شود. به بیان دیگر با انجام محاسبات اولیه و پیش‌پردازش‌ها در نزدیکی محل تولید داده‌ها، میزان ترافیک ارسالی به سرور اصلی و در نتیجه تاخیر دسترسی به آن کاهش می‌یابد. لذا این رویکرد می‌تواند در بهبود عملکرد یادگیری فدرال نیز موثر باشد و این موضوع اصلی این پروژه را تشکیل می‌دهد.

\vspace*{0.5 cm}

\noindent\textbf{واژه‌های کلیدی:}
1- یادگیری فدرال 2- پردازش لبه 3- اینترنت اشیاء 4-يادگيری ماشين.
\end{small}